\documentclass[conference]{IEEEtran} % Classe padrão para conferências SBC

\usepackage[utf8]{inputenc} % Codificação de caracteres
\usepackage[brazil]{babel} % Idioma em português
\usepackage{amsmath} % Pacote para fórmulas matemáticas
\usepackage{graphicx} % Pacote para inserir imagens

\title{Jogo da velha por sockets TCP/IP} % Título do projeto
\author{\IEEEauthorblockN{Guilherme Augusto de Oliveira Soares} \\
\IEEEauthorblockA{Universidade Federal do Mato Grosso do Sul} \\
\{g.augusto@ufms.br\}} % Coloque o seu nome, sua instituição e e-mail

\begin{document}

\maketitle % Gera o título e os autores

\begin{abstract}
Este relatório descreve o desenvolvimento de um jogo da velha (Tic-Tac-Toe) utilizando sockets TCP/IP, com o objetivo de permitir que dois jogadores interajam em rede.
\end{abstract}

\section{Introdução}

 Um jogo da velha implementado em Python onde oferece uma experiência de jogo no console, onde os jogadores fazem suas jogadas em um tabuleiro 3x3. A comunicação entre os jogadores é realizada via sockets, permitindo que um servidor gerencie a partida e dois clientes participem de forma independente. O projeto também oferece funcionalidades como exibição do tabuleiro, verificação de vitórias e reinício do jogo, além de validação de movimentos inválidos. Este trabalho explora o uso de redes para criar uma aplicação simples, porém eficaz, utilizando programação orientada a sockets.

\section{Como instalar e rodar o projeto}

Para instalar e rodar o projeto, siga os seguintes passos:

\begin{enumerate}
    \item Clone o repositório do projeto:
    \begin{verbatim}
    git clone https://github.com/guisoaresdev/tcp-tictactoe-py
    \end{verbatim}
    
    \item Acesse o diretório do projeto:
    \begin{verbatim}
    cd tcp-tictactoe-py
    \end{verbatim}
    
    \item Para rodar o servidor:
    \begin{verbatim}
    python host.py
    \end{verbatim}

    \item Para rodar os clientes (dois terminais):
    \begin{verbatim}
    python client.py
    \end{verbatim}
\end{enumerate}
\section{Funcionalidades}

O sistema oferece as seguintes funcionalidades principais:

\begin{itemize}
    \item \textbf{Funcionalidade 1}: Exibir o tabuleiro quando os dois jogadores estão conectados.
    \item \textbf{Funcionalidade 2}: Quando um jogador ganhar, questiona se eles desejam uma revanche.
    \item \textbf{Funcionalidade 3}: Quando um jogador insere uma posicão inválida, é requisitado um novo
 movimento.
\end{itemize}

\begin{thebibliography}{1}
\bibitem{TicTacToe Python Game} GrantWilliams99, TicTacToe, 2021, https://github.com/GrantWilliams99/TicTacToe. 
\end{thebibliography}

\end{document}
